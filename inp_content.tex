\def \currentAuthor {Gabi Sorglos} %so kann jederzeit der Autor geändert werden -> wird in der Fusszeile angezeigt.

\chapter*{Einleitende Bemerkungen}

\chapter*{Notationen}
Beschreibung wie Code, Hinweise, Zitate etc. formatiert werden  

\chapter{Projektmanagement}

\section{Metainformationen}
\subsection{Team}
Christopher Haider (Leiter, Programmierer, Design), Maximilian Egger ()
\subsection{Betreuer}
NEUNER Dominik, MSc
\subsection{Partner}
HAK Imst
\subsection{Ansprechpartner}
NEUNER Dominik, MSc
\section{Vorerhebungen}
\subsection{Projektzieleplan}

\textbf {-Einteilung:}


\textbf {OBERZIEL} Es soll eine Applikation entwickelt werden, bei der Fragen (zu einem Thema) und zugehörigen mögliche Antworten (mit einer richtigen) erstellt werden können. Jeder Frage muss ein Standort auf der Karte zugewiesen werden.
Im Front-End können Fragen im Umkreis aufgelistet und gelöst werden. Für jede gelöste Frage erhöht sich der Punktestand des Benutzers.  
.


\textbf {LEISTUNGSZIEL} Das Projekt wird mehrere Mitarbeiter beinhalten und wir versuchen den einzelnen Mitarbeitern die optimale Arbeit zuzuweisen, um eine pünktliche Fertigstellung mit würdiger Qualität zu erreichen. Implementierung von aktuellen Karten und Standort, kleine Übersicht des Punktestandes, Backend/Frontend;


\textbf {KOSTENZIEL} Die Kosten belaufen sich ca. in Höhe von 8.500 - 12.500 € abhängig von der Zeit der aktiven Arbeit jedes einzelnen Mitarbeiters/Programmierer plus den Überstunden inklusive allen Aufwänden für die Programmierung, Serverwartungen, laufende Aktualisierungen von den Daten. Abgesehen von den Steuern.


\textbf {TERMINZIEL} Dieses etwas umfangreichere Projekt sollte mit Einplanung vom 14.03.2017 in einem Zeitrahmen bis Ende November/Dezember in Form von einer kompakten App im Store verfügbar sein.


\textbf {DETAILZIEL} Unserem Partner/Kunden vor der endgültigen Fertigstellung des Projekts eine Testapp /  Prototypapp vorzeigen und nochmals eine endgültige Absprache mit den Kunden abhalten.



\subsection{Projektumfeld}

\begin{figure}
	\centering
	\includegraphics[width=1\linewidth]{../Grafik}
	\caption{}
	\label{fig:grafik}
\end{figure}


\begin{figure}
	\centering
	\includegraphics[width=1\linewidth]{../Grafik1}
	\caption{}
	\label{fig:grafik1}
\end{figure}






Welche Bedrohungen bestehen für das Projekt
•	Budget Überschreitung
•	Termine nicht einhalten
•	Auftragsgeber sagt ab
•	Auftragsgeber ändert seine Anforderungen
•	Probleme bei der Realisierung






\subsection{Risikoanalyse}
\begin{figure}
	\centering
	\includegraphics[width=1.2\linewidth]{../Grafik2}
	\caption{}
	\label{fig:grafik2}
\end{figure}

\section{Pflichtenheft}
\subsection{Zielbestimmung}
\begin{itemize}
	\item Projektbeschreibung
	\item IST-Zustand
	\item SOLL-Zustand
	\item NICHT-Ziele (Abgrenzungskriterien)
\end{itemize}
\subsection{Produkteinsatz und Umgebung}
\begin{itemize}
	\item Anwendungsgebiet
	\item Zielgruppen
	\item Betriebsbedingungen
	\item Hard-/Softwareumgebung
\end{itemize}
\subsection{Funktionalitäten}
\begin{itemize}
	\item MUSS-Anforderungen
	\begin{itemize}
		\item Funktional
		\item Nicht-funktional
	\end{itemize}
	\item KANN-Anforderungen
	\begin{itemize}
		\item Funktional
		\item Nicht-funktional
	\end{itemize}
\end{itemize}
\subsection{Testszenarien und Testfälle}
\begin{itemize}
	\item Beschreibung der Testmethodik
	\item Testfall 1
	\item Testfall 2
	\item \ldots
\end{itemize}
\subsection{Liefervereinbarung}
\begin{itemize}
	\item Lieferumfang
	\item Modus
	\item Verteilung(Deployment)
\end{itemize}
\section{Planung}
\subsection{Projektstrukturplan}
\subsection{Meilensteine}
\subsection{Gant-Chart}
\subsection{Abnahmekriterien}
\subsection{Pläne zur Evaluierung}
\subsection{Ergänzungen und zu klärende Punkte}

\chapter{Vorstellung des Produktes}
Vorstellung des fertigen Produktes anhand von Screenshots, Bildern, Erklärungen.

\chapter{Eingesetzte Technologien}
\begin{itemize}
	\item Kurzbeschreibung aller Technologien, die verwendet wurden.
	\item Technologien die aus dem Unterricht bekannt sind, nur nennen und deren  Einsatzzweck im Projekt beschreiben, nicht die Technologien selbst.
	\item Technologien die aus dem Unterricht nicht bekannt sind, im Detail beschreiben incl. deren Einsatz im Projekt
	\item Fokus aus eingesetzten Frameworks
\end{itemize}

\chapter{Problemanalyse}
\section{USE-Case-Analyse}
\begin{itemize}
	\item UseCases auf Basis von Benutzerzielen identifizieren: 
	\begin{itemize}
		\item Benutzer eines Systems identifizieren
		\item Benutzerziele identifizieren (Interviews)
		\item Use-Case-Liste pro Benutzer definieren
	\end{itemize}
	\item UseCases auf Basis von Ereignissen identifizieren: 
	\begin{itemize}
		\item Externes Event triggert einen Prozess
		\item zeitliches Event triggert einen Prozess (Zeitpunkt wird erreicht) 
		\item State-Event (Zustandsänderung im System triggert einen Prozess)
	\end{itemize}
	\item Werkzeuge:
	\begin{itemize}
		\item USE-Case-Beschreibungen (textuell, tabellarisch)
		\item USE-Case-Diagramm
		\item Aktivitätsdiagramm für den Use-Case (Interaktion zwischen Akteur und System abbilden)
		\item System-Sequenzdiagramm (Spezialfall eines Sequenzdiagramms: Nur 1 Akteur und 1 Objekt, das Objekt ist das komplette System, es geht um die Input/Output Requirements, die abzubilden sind)
	\end{itemize}
\end{itemize}

\section{Domain-Class-Modelling}
\begin{itemize}
	\item "Dinge" (Rollen, Einheiten, Geräte, Events etc.) identifizieren, um die es im Projekt geht
	\item ER-Modellierung oder Klassendiagramme
	\item Zustandsdiagramme (zur Darstellung des Lebenszyklus von Domain-Klassen darstellen)
\end{itemize}

\section{User-Interface-Design}
\begin{itemize}
	\item Mockups
	\item Wireframes
\end{itemize}


\chapter{Systementwurf}

\section{Architektur}

\subsection{Design der Komponenten}

Darstellung und Beschreibung der Systemarchitektur;

\begin{itemize}
	\item  statische Zerlegung des Systems in seine physischen Bestandteile (Komponenten, Komponentendiagramm)
	\item (textuelle) Beschreibung des dynamischen Zusammenwirkens aller Komponenten 
	\item (textuelle) Beschreibung der Strategie für die Architektur, d. h. wie die Architektur in Statik und Dynamik funktionieren soll.
	\item Verwendung von Referenzarchitekturen bzw. Architekturmustern (als Schablonen, z.B. MVC. Plugin, Pipes and Filters)
	\begin{itemize}
		\item MVC
		\item Schichten
		\item Pipes
		\item Request Broker
		\item Service-Oriented
	\end{itemize}
\end{itemize}

\subsection{Benutzerschnittstellen} 
\begin{itemize}
	\item Design des UIs
	\item Dialoge, Dialogsteuerung, Ergonomie, Gestaltung, Eingabeüberprüfungen
\end{itemize}

\subsection{Datenhaltunskonzept}
\begin{itemize}
	\item Design der Datenbank (ER-Modell)
	\item Design des Zugriffs auf diese Daten (Datenhaltungskonzept)
	\item Caching, Transaktionen
\end{itemize}

\subsection{Konzept für Ausnahmebehandlung}
\begin{itemize}
	\item Systemweite Festlegung, wie mit Exceptions umgegangen wird
	\item Exceptions sind primär aus den Bereichen UI, Persistenz, Workflow-Management
\end{itemize}

\subsection{Sicherheitskonzept}
Beschreibung aller sicherheitsrelevanten Designentscheidungen

\begin{itemize}
	\item Design der Security-Elemente
	\item Design von Safety-Elementen (Fehlertoleranz, Verfügbarkeit etc.)
\end{itemize}

\subsection{Design der Testumgebung}
\begin{itemize}
	\item wie wird getestet (Unit-Testing, Integrationstesting, Systemtests, Akzeptanztests)
	\item Testumgebung, Testprozess, Teststrategie, Testmethoden, Testfälle
\end{itemize}


\subsection{Desing der Ausführungsumgebung}
\begin{itemize}
	\item Deployment (DevOps)
	\item Betrieb (besonders Hoch- und Hertunerfahren der Anwendung)
\end{itemize}

\section{Detailentwurf}

Design jedes einzelnen USE-Cases

\begin{itemize}
	\item Design-Klassendiagramme vom Domain-Klassendiagramm ableiten (incl. detaillierter Darstellung und Verwendung von Vererbungshierarchichen, abstrakten Klassen, Interfaces)
	\item Sequenzdiagramme vom System-Sequenz-Diagramm ableiten
	\item Aktivitätsdiagramme
	\item Detaillierte Zustandsdiagramme für wichtige Klassen
\end{itemize}

Verwendung von CRC-Cards (Class, Responsibilities, Collaboration) für die Klassen
\begin{itemize}
	\item um Verantwortlichkeiten und Zusammenarbeit zwischen Klassen zu definieren und
	\item um auf den Entwurf der Geschäftslogik zu fokussieren
\end{itemize}

Design-Klassen für jeden einzelnen USE-Case können z.B. sein:

\begin{itemize}
	\item UI-Klassen
	\item Data-Access-Klassen
	\item Entity-Klassen (Domain-Klassen)
	\item Controller-Klassen
	\item Business-Logik-Klassen
	\item View-Klassen
\end{itemize}

Optimierung des Entwurfs (Modularisierung, Erweiterbarkeit, Lesbarkeit):

\begin{itemize}
	\item Kopplung optimieren
	\item Kohäsion optimieren
	\item SOLID
	\item Entwurfsmuster einsetzen
\end{itemize}

\chapter{Implementierung}
Detaillierte Beschreibung der Implementierung aller Teilkomponenten der Software entlang der zentralsten Use-Cases:

\begin{itemize}
	\item GUI-Implementierung
	\item Controllerlogik
	\item Geschäftslogik
	\item Datenbankzugriffe
\end{itemize}

Detaillierte Beschreibung der Teststrategie (Testdriven Development):

\begin{itemize}
	\item UNIT-Tests (Funktional)
	\item Integrationstests
\end{itemize}

Zu Codesequenzen:
\begin{itemize}
	\item kurze Codesequenzen direkt im Text (mit Zeilnnummern auf die man in der Beschreibung verweisen kann)
	\item lange Codesequenzen in den Anhang (mit Zeilennummer) und darauf verweisen (wie z.B. hier \cref{qj})
\end{itemize}

\chapter{Deployment}
\begin{itemize}
	\item Umsetzung der Ausführungsumgebung
	\item Deployment
	\item DevOps-Thema
\end{itemize}

\chapter{Tests}

\section{Systemtests} 
Systemtests aller implementierten Funktionalitäten lt. Pflichtenheft
\begin{itemize}
	\item Beschreibung der Teststrategie
	\item Testfall 1
	\item Testfall 2
	\item Tesfall 3
	\item …
\end{itemize}

\section{Akzeptanztests}

\chapter{Projektevaluation}
siehe Projektmanagement-Unterricht

\chapter{Benutzerhandbuch} 
falls im Projekt gefordert

\chapter{Betriebswirtschaftlicher Kontext}
BW-Teil

\chapter{Zusammenfassung}
\begin{itemize}
	\item Etwas längere Form des Abstracts
	\item Detaillierte Beschreibung des Outputs der Arbeit
\end{itemize}