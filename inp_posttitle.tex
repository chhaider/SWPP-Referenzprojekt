\chapter*{Eidesstattliche Erklärung}
Ich erkläre an Eides statt, dass ich die vorliegende Diplomarbeit selbst verfasst und keine anderen als die angeführten Behelfe verwendet habe. Alle Stellen, die wörtlich oder inhaltlich den angegebenen Quellen entnommen wurden, sind als solche kenntlich gemacht.
Ich bin damit einverstanden, dass meine Arbeit öffentlich zugänglich gemacht wird.

\vspace{1cm}
\begin{tabular}{c c c}
	& \hspace{4cm} & \\\cline{1-1}
	Ort, Datum & & \\
	\vspace{2cm}
	& & \\\cline{1-1}\cline{3-3}
	Christopher Haider & & Maximilian Egger \\ 
	\vspace{2cm}
	& & \\\cline{1-1}\cline{3-3}
	Dominik Neuner & & Harald Sommer \\ 
\end{tabular}

\chapter*{Abnahmeerklärung}
Hiermit bestätigt der Auftraggeber, dass das übergebene Produkt dieser Diplomarbeit den dokumentierten Vorgaben entspricht. Des Weiteren verzichtet der Auftraggeber auf unentgeltliche Wartung und Weiterentwicklung des Produktes durch die Projektmitglieder bzw. die Schule.

\vspace{1cm}
\begin{tabular}{c}
	\\\cline{1-1}
	Ort, Datum\\
	\vspace{2cm}
	\\\cline{1-1}
	Auftraggeber
\end{tabular}	

\chapter*{Vorwort}
z. B. Hinweise, wie das bearbeitete Thema gefunden wurde oder Dank für die Betreuung (Kooperationspartner/in, Betreuer/innen, Sponsoren) etc.


\chapter*{Abstract (Deutsch)}
Es soll eine Applikation entwickelt werden, bei der Fragen (zu einem Thema) und zugehörigen mögliche Antworten (mit einer richtigen) erstellt werden können. Jeder Frage muss ein Standort auf der Karte zugewiesen werden. 
Im Front-End können Fragen im Umkreis aufgelistet und gelöst werden. Für jede gelöste Frage erhöht sich der Punktestand des Benutzers.  


\chapter*{Abstract (Englisch)}
An App should  be developed, where questions (to a topic) and the belonging answers (with one correct)  are able to be created. Each Question must have a location on the map. In the front-end-mode questions in permiter can be listed and solved. After every solved Question the score of the user will be increased.